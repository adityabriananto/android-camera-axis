	\section{Les formats vidéo}
		\subsection{MPEG-4}
		\subsubsection{Définition}
		Le \textit{MPEG-4} est une norme de codage vidéo. Celui-ci
		permet de gérer toutes les nouvelles applications multimédias comme le
		téléchargement et le streaming sur Internet, le multimédia sur téléphone
		mobile, etc.
		\subsubsection{Avantages}
		\begin{itemize}
		  \item L'avantage principal du MPEG-4 est qu'il permet de s'adapter à beaucoup de
		supports, tels que ceux cités ci-dessus.
		  \item Il permet également de transmettre le son avec la vidéo.
		\end{itemize}
		\subsubsection{Inconvénient}
		\begin{itemize}
		  \item L'inconvénient majeur est son utilisation impossible avec HTTP.
		\end{itemize}
		
		\subsection{M-JPEG ou Motion JPEG}
		\subsubsection{Définition}
		Le \textit{M-JPEG} est un codec vidéo qui compresse les images une à une en
		JPEG.
		\subsubsection{Avantages}
		\begin{itemize}
		  \item Il permet une utilisation avec HTTP.
		  \item La compression des images est plus rapide que le codage MPEG-4.
		\end{itemize}
		
		\subsubsection{Inconvénient}
		\begin{itemize}
		  \item Le principal inconvénient est l'absence de transmission du son.
		  \newline
		\end{itemize}
		
	\section{Les formats audio}
		\subsection{G.711}
			Le \textit{G.711} est une norme de compression audio.
			\begin{itemize}
			  \item Échantillonnage : 8000 Hz 
			  \item Bande passante sur le réseau : 64 ou 56 kbit/s
			  \item Type de codage : Modulation d'impulsion codée (MIC).
			\end{itemize}
		\subsection{G.726}
			Le \textit{G.726} est une norme de compression audio.
			\begin{itemize}
			  	\item Bande passante sur le réseau : 16, 24, 32 ou 40 kbit/s
				\item Type de codage : Modulation par impulsions et codage différentiel
				adaptatif (MICDA).
				\end{itemize}
		\subsection{AAC}
			L'\textit{AAC} (Advance Audio Coding) est un algorithme de compression audio
			qui a pour but de réduite la qualité, pour offrir un meilleur débit binaire.
\newline