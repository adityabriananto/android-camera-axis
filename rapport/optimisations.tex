\chapter{Optimisations / Extensions futures}
Malgrès que le projet semble fonctionnel, il reste tout de même de nombreuses
améliorations et optimisations à apporter. 

\subsubsection{Enregistrement vidéo et retransmission du signal audio}
Actuelement notre projet ne permet pas l'enregistrement de vidéo dut
principalement à l'utilisation du format \textit{MJPEG}. En effet, le seul moyen
pour nous d'enregistrer la vidéo serai la sauvegarde successive des images recu.
Un solution plus performente serait de surcharger un partie ou l'ensemble des
classes permettant l'utilisation du protocole \textit{rtsp} afin d'y integrer un
mécanisme d'authentification pour permettre l'utilisation de la vidéo
\textit{mpeg4}. Ainsi nous pourrions remplacer le player \textit{mjpegView} par
le media player proposé par android, afin de bénéficier de ses nombreux
avantages (lecture, pause, enregistrement grace au \textit{mediaRecorder},
retransmission du son grace au format \textit{mpeg4} \ldots).

\subsubsection{Édition des fenêtres de détection  de mouvement}
Nous offrons a l'utilisateur la possibilité de détécter les mouvement sur une
fenetre au prealablement dessinée sur la vidéo. Cependant actuelement, le seul
moyen d'editer cette fenetre est d'arreter la détéction puis de relancer
celle-ci avec la nouvelle fenetre. En effet, meme si l'utilisateur modifie la
fenetre durant la detection, nous n'avons pas eu le temps d'implémenter la mise
à jour de sa position aupres de la caméra.\newline
Pour ce faire, l'idée serait d'implementer une extension de notre classe
\textit{drawRectOnTouchView} plus adapté à notre utilisation. Par exemple, la
surcharge de la fonction \textit{onTouchEvent()} nous permetterait d'ajouter une
mise a jour de la caméra (passé en parametre lors de sa construction) comme le
montre l'esquisse de code suivante : 
\begin{lstlisting}[format=java]
public class drawRectOnTouchViewForCamera extends drawRectOnTouchView {
private CameraControl cam;
private boolean isDetectionRunning;
public drawRectOnTouchView(Context context, AttributeSet attrs, camera c,
boolean etat) { super(context,attrs);
this.cam= new CameraControl(context,c);
this.isDetectionRunning = stat;
}


public void updateWindow(drawRectOnTouchView d){
  int absoluteTop = (int) (getStart().y * 10000 / d.getBottom());
  int absoluteBottom = (int) (getEnd().y * 10000 / d.getBottom());
  int absoluteRight = (int) (getEnd().x * 10000 / d.getRight());
  int absoluteLeft = (int) (getStart().x * 10000 / d.getRight());
  camC.updateMotionDParam("Top", ""
                        + absoluteTop);
  camC.updateMotionDParam("Bottom", ""
                        + absoluteBottom);
  camC.updateMotionDParam("Right", ""
                        + absoluteRight);
  camC.updateMotionDParam("Left", ""
                        + absoluteLeft);
               }         
                        

@Override
public boolean onTouchEvent(MotionEvent event) {
/* Recuperer et trace le rectangle comme le defini la classe drawRectOnTouchView
actuelement utilisee */
 super(event);
 switch (event.getAction()) {
    case MotionEvent.ACTION_DOWN:
        break;
    case MotionEvent.ACTION_MOVE:
        break;
    case MotionEvent.ACTION_UP:
        if(isDetectionRunning){
updateWindow(this);
        }
        break;
    }
    return true;
    }
}
\end{lstlisting}
Ainsi grace à la classe \textit{drawRectOnTouchViewForCamera}, à chaque fois que
l'utilisateur mettre a jours la fenetre, celle-ci sera directement (à plus ou
moins la latence réseau) mise a jour sur la caméra.

\subsubsection{Ajout d'évènements suite à la détection (mail / snapshot)}
Lorsque qu'un mouvement est détecté par la caméra, notre notification consiste a
prendre un cliché instantanné, puis d'emettre une vibration de l'appareil. Il
sera pertinant d'ajouter à l'utilisateur la possibilité de regler le type de
notification. Ceci pourrait etre effectuer simplement par l'activation ou la
desactivation de ces traitements dans le \textit{SharedPreference} deja
implémenté. \newline
Nous pourrions utiliser la meme methode que celle utilisé pour l'activation ou
la desactivation de la fenetre d'astuce. Il suffirait ensuite, dans le handler
du service de detection de mouvement, de verifier l'etat (activer ou desactiver)
des actions implémentées.

\subsubsection{Compte utilisateur}
La Caméra propose de nombreuses fonctions permettant de l'administrer a
distance. Par exemple l'envoie d'une requete \textit{GET} à l'adresse 
\begin{lstlisting}
http://myserver/axis-cgi/admin/pwdgrp.cgi?
action=add&user=joe&pwd=foo&grp=axuser&sgrp=axadmin:axoper:axview&comment=Joe
\end{lstlisting}
ajoute l'utilisateur joe avec le mot de pass foo. (Exemple tiré de la
documentation officiel).\newline
Parmis les principales fonctions d'adninistration disponible, on retrouve : 
\begin{itemize}
  \item Ajout-Suppression d'utilisateurs/groupes
  \item Reset factory
  \item Backup-Restore
  \item Redemarrage du serveur
  \item \ldots
\end{itemize}
On peut conclure qu'il existe encore un grand nombre de service fourni par la
caméra à implémenter, même si la pluspart d'entre eux ne necessite aucunes
modification majeur.
