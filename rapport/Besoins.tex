\chapter{Analyse des besoins}
Notre application a pour but principal d'afficher le flux d'une caméra distante (de marque Axis) et de pouvoir la contrôler directement à partir du périphérique basé sur Android.
Il a fallu d'abord définir les besoins essentiels à la réalisation d'une telle application aux niveaux technique et fonctionnel, puis ensuite s'attacher à rendre cette application
performante, fiable et ergonomique.

\section{Besoins fonctionnels}
Les besoins suivants sont nécessaires au fonctionnement de l'application, cependant certains sont plus importants que d'autres et donc prioritaires pour arriver à la finalité du projet.
\begin{itemize}
  \item \textbf{Flux vidéo} : l'application doit retransmettre le flux vidéo et
  audio de la caméra d'une manière fluide et sans interruption.
  \item \textbf{Multivue} : elle doit proposer le choix d'afficher une seule vue en
  plein écran ou plusieurs vues simultanément sur le même écran.
  \item \textbf{Liste de caméras} : il doit pouvoir gérer une liste de caméras avec
  différentes actions d'ajout, de modification et de suppression d'entités.
  \item \textbf{Import/Export} : la liste de caméras doit pouvoir être importée ou
  exportée afin de la rendre portable.
  \item {Contrôle PTZ tactile} : l'utilisateur doit pouvoir contrôler les
  propriétés Pan/Tilt/Zoom de la caméra tactilement d'une manière fluide.
  \item \textbf{Fonctions avancées} : les fonctionnalités spécifiques de la caméra
  doivent pouvoir être utilisées (comme l'autofocus ou l'auto-iris par exemple).
  \item \textbf{Capture} : une capture d'écran doit pouvoir être réalisée (avec
  choix de la résolution) et enregistrée sur le périphérique.
  \item \textbf{Détection de mouvements} : la détection de mouvements doit pouvoir
  être activée/désactivée et gardée en tâche de fond avec les différentes fenêtres de détection choisies
  \item \textbf{Réglage des propriétés} : l'utilisateur doit pouvoir régler les
  propriétés générales de l'application (comme le temps de réponse, le taux de rafraîchissement ou encore la sensibilité) et les propriétés spécifiques à la détection de mouvements
  \item \textbf{Notifications} : un système de notifications doit prévenir
  l'utilisateur du succès de la capture d'écran et de la présence d'un mouvement détecté
  \item \textbf{QrCode} : une caméra doit pouvoir être ajoutée à partir de la
  lecture d'un QrCode.
  \item \textbf{Réseau} : l'application doit s'adapter au réseau disponible Wifi/3G
  par le choix de la résolution de l'image reçue.
\end{itemize}

\section{Besoins non fonctionnels}
L'application ne serait pas vraiment performante si certains aspects non liés au fonctionnement n'étaient pas pris en compte. Nous avons donc tenu compte des besoins suivants pour développer un produit final Android intéressant.
\begin{itemize}
  \item \textbf{Performances} : l'application doit avoir de bonnes performances au
  niveau transmission et une vitesse de rafraichissement satisfaisante pour permettre la surveillance directe.
  \item \textbf{Sûreté d'exécution} : l'exécution doit être sûre en cas de faible
  connectivité et ne pas terminer brusquement.
  \item \textbf{Réactivité de contrôle} : la vitesse de réponse doit être la plus
  faible possible pour proposer un contrôle tactile de la caméra réactif.
  \item \textbf{Fiabilité de sauvegarde} : les données sauvegardées par
  l'application doivent être intactes et retrouvées sans problèmes.
  \item \textbf{Ergonomie} : l'application doit offrir une bonne ergonomie, une
  interaction avec l'utilisateur intuitive et être facile d'utilisation pour tous publics.
  \item \textbf{Réactivité de détection} : le temps entre le mouvement détecté et
  la notification d'alerte résultante doit être le plus court possible.
  \item \textbf{Compatibilité} : l'application doit être compatible avec les
  différents modèles de caméras Axis et s'adapter aux fonctionnalités variables.
\end{itemize}
L'ensemble de ces besoins ont été implémenté, à l'exception de la retransmission
du son. En effet les codecs utilisé par notre caméra (G.711 ,G.726, AAC) ne sont
pas implémentés dans les versions d'android que nous disposons. Seul le codec AAC fait son
apparition à partir de la version 3.0 d'android.
\clearpage