\documentclass{beamer}
\usepackage[utf8]{inputenc}
\usepackage{xcolor}
\usepackage{listings}
\usepackage{graphicx}
\usetheme[secheader]{Boadilla}
\definecolor{coultitre}{rgb}{0.2,0.55,0}
\setbeamercolor{structure}{fg=coultitre, bg=coultitre!40} 





\title{Vidéo surveillance, Streaming vidéo et contrôle de caméra via Android }
\author{Jérôme NAHELOU, Quentin NEBOUT, Romain SOLVE, Fabien QUINTARD}
\institute{\large{Chargé de Projet : David BROMBERG}\\ \bigskip{}
\small{Université Bordeaux 1}}
\date{29 mars 2010}


\begin{document}
\frame[plain]{\titlepage}

\AtBeginSection[]{
\begin{frame}<beamer>
\frametitle{Plan}
\tableofcontents[currentsection]
\end{frame}}

\begin{frame}
\frametitle{Plan de l'exposé}
\tableofcontents
\end{frame}

\section{Introduction}
  \begin{frame}
   \frametitle{Description}
  Introduction
  % schéma projet : dessin camera + portablets et tablettes android en réseau %
   \begin{itemize}
    \item<2-> Android: OS pour appareil mobile, basé sur noyaux linux
    \item<3-> Choix: M-JPEG, HTTP-GET
   \end{itemize}
  \end{frame}
  
  \begin{frame}
   \frametitle{Besoins}
  Introduction
  % schéma besoins F / besoins NF %
  \end{frame}

\section{Aspect général de l'application}
  \subsection{Description}
  \begin{frame}
   \frametitle{Description}
  Aspect général de l'application convivial
   \begin{itemize}
    \item<2-> Astuces pour mieux connaître les fonctionnalités
    \item<3-> Application Multi-langue avec détection automatique
   \end{itemize}
  \end{frame}

\section{Simple vue}
\subsection{Spécifications}
 \begin{frame}
   \frametitle{Spécifications}
   \begin{itemize}
    \item<2-> Réutilisation d'un lecteur MJPEG existant : MjpegView
    \item<3-> Interface de contrôle de la caméra
    % screen simple vue %
    \end{itemize}
\end{frame}

\section{Multi-vue}
\subsection{Spécifications}
 \begin{frame}
   \frametitle{Spécifications}
   \begin{itemize}
    \item<2-> Implémentation de layouts personnalisés pour 2 à 6 caméras
    % schéma ou screen multi-vue %
    \item<3-> Rafraîchissement multi-threadé (un thread par caméra)
    % schéma UML de Jérôme ;) %
    \item<4-> Ajout de listeners pour gérer les caméras et passer en simple vue 
    \end{itemize}
\end{frame}

\section{Contrôle de la caméra}
\subsection{Implémentation}
 \begin{frame}
   \frametitle{Implémentation}
   \begin{itemize}
    \item<2-> Communication avec la caméra par requêtes HTTP
    \item<3-> Chargement de la configuration à l'initialisation de la vidéo
    \item<4-> Méthode de construction et d'envoi des requêtes avec
    authentification et délai variable de connexion
   \end{itemize}
\end{frame}

\subsection{Interfaces}
 \begin{frame}
\frametitle{Interfaces}
   Plusieurs interfaces de contrôle direct :
   \begin{itemize}
    \item<2-> Flux vidéo en continu pour actions en temps réel
    \item<3-> Contrôle du Pan/Tilt/Zoom de manière tactile
    % schéma notion tactile %
    \item<4-> Interface de commandes avancées adaptée à la caméra
    \item<5-> Module de gestion de détection de mouvements
   \end{itemize}
\end{frame}

\section{Détection de mouvements}
\subsection{Principe Axis}
 \begin{frame}
   \frametitle{Pricipe Axis}
    Le mécanisme de la détection de mouvements par Axis est le suivant :
   \begin{itemize}
    \item<2-> Ajout d'une fenêtre de détection spécifique (coordonnées,
    sensibilité, \ldots) par l'utilisateur
    \item<3-> Solution 1 : Ajout d'un événement déclenché automatiquement à la
    présence d'un mouvement (email, SMS, \ldots)
    \item<4-> Solution 2 : Demande du flux de niveaux de détection pour un
    calcul et un traitement personnalisés
   \end{itemize}
\end{frame}

\subsection{Implémentation Android}
 \begin{frame}
   \frametitle{Implémentation Android}
    La solution 2 a été retenue pour notre application :
    \begin{itemize}
    \item<2-> Définition de la fenêtre directement sur l'écran (composant
    graphique)
    \item<4-> Mise en arrière-plan du thread de calcul (utilisation d'un service
    Android)
    \item<5-> Alerte par notification : vibration + snapshot
    % deuxième schéma UML de Jérôme %
   \end{itemize}
\end{frame}

\section{Tests Unitaires}
\subsection{Principe}
 \begin{frame}
   \frametitle{Principe}
Le but des tests unitaires est de:
\begin{itemize}
    \item<2-> Garantir le bon fonctionnement de l'application
    \item<3-> Vérifier si les fonctions terminent bien
    \item<4-> Identifier étape par étape les erreurs éventuelles
   \end{itemize}
\end{frame}

\subsection{Tests en réel}
 \begin{frame}
   \frametitle{Tests en réel}
\begin{itemize}
    \item<2-> Suivi de l'exécution en continu avec le LogCat d'Android
    \item<3-> Utilisation intensive de 3 téléphones de marques différentes
    \item<4-> Tests sur une caméra Axis PTZ 214 avec l'ensemble des
    fonctionnalités implémentées
   \end{itemize}
\end{frame}

\section{Conclusion}
\subsection{Optimisations}
 \begin{frame}
   \frametitle{Optimisations}

\end{frame}

\end{document}